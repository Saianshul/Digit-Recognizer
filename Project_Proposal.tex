\documentclass{article}
\usepackage[utf8]{inputenc}
\usepackage{amsmath}
\usepackage{amssymb}
\usepackage{xcolor} % for color definitions
\usepackage{mdframed} % for framing and shading the problems
\usepackage{lipsum} % for generating text, remove in your actual document
\usepackage{geometry} % for page layout
\usepackage{titling} % for title page layout

% Set the page size and margins
\geometry{letterpaper, portrait, margin=1in}

% Define a new environment for the problems that takes one argument for the problem number
\newenvironment{problem}[1]{
    \begin{mdframed}[backgroundcolor=gray!20, skipabove=\baselineskip, skipbelow=\baselineskip, nobreak=true, innerleftmargin=10pt, innerrightmargin=10pt, innertopmargin=10pt, innerbottommargin=10pt]
    \textbf{Problem #1.}
}{
    \end{mdframed}
}

% Define a new environment for the proofs
\newenvironment{proof}{
    \begin{mdframed}[nobreak=true, innerleftmargin=10pt, innerrightmargin=10pt, innertopmargin=10pt, innerbottommargin=10pt]
    \textbf{Proof.}
}{
    \hfill $\square$
    \end{mdframed}
}

% Remove section numbering
\makeatletter
\renewcommand{\@seccntformat}[1]{}
\makeatother

% Remove table of contents numbering
\renewcommand{\thesection}{}

% Title page info
\title{Project Proposal \\ \large Capstone: The Art of Approximation}
\author{Ojas Chaturvedi, Zaheen Jamil, Saianshul Vishnubhaktula, Ritwik Jayaraman}
\date{November 21, 2023}

% ------------------------------------------------------------------------------
\begin{document}

% Title page
\begin{titlingpage}
    \maketitle
    \tableofcontents
\end{titlingpage}

% ------------------------------------------------------------------------------

\newpage \section{Idea 2: AI-Based Precision Medicine Platform}
    \textbf{Purpose:} Build an AI-powered platform that can analyze genetic data, medical records, and research to provide personalized treatment recommendations. \\ \\
    \textbf{Member 1: Bioinformatics Scientist}
        \begin{itemize}
            \item Develop algorithms for processing and analyzing large genomic datasets.
            \item Work on integrating genetic data with patient electronic health records (EHR).
            \item Lead the research on disease markers and drug response prediction.
        \end{itemize}
    \textbf{Member 2: Machine Learning Engineer}
        \begin{itemize}
            \item Build predictive models for disease risk and treatment outcomes.
            \item Implement natural language processing to extract insights from medical literature.
            \item Ensure the interpretability and fairness of AI models.
        \end{itemize}
    \textbf{Member 3: Data Security Expert}
        \begin{itemize}
            \item Design a secure infrastructure to protect sensitive health data.
            \item Ensure compliance with HIPAA and other health data protection regulations.
            \item Implement robust access controls and audit trails.
        \end{itemize}
    \textbf{Member 4: Clinical Specialist and User Experience Designer}
        \begin{itemize}
            \item Provide clinical expertise to guide the platform's development.
            \item Design the user interface for clinicians and patients, focusing on usability.
            \item Facilitate clinical trials and collect user feedback for continuous improvement.
        \end{itemize}

\newpage \section{Proposal}
    \begin{enumerate}
        \item \textbf{Language:} Python, a simple and popular language for machine learning and data science due to its extensive libraries and frameworks.
        \item \textbf{Objective:} Build an AI-powered platform that can analyze genetic data, medical records, and research to provide personalized treatment recommendations.
        \item \textbf{Implementation:}
            \begin{enumerate}
                \item \textbf{Overview of steps:}
                    \begin{enumerate}
                        \item \textbf{Data Collection:} Collect genetic data, medical records, and research data from various sources and databases.
                        \item \textbf{Data Processing:} Process the data to extract relevant features.
                        \begin{enumerate}
                            \item \textbf{Homomorphic Encryption} will be utilized to protect sensitive health data instead of having to conform to HIPAA and other health data protection regulations.
                        \end{enumerate}
                        \item \textbf{Model Training:} Train machine learning models to predict disease risk and treatment outcomes.
                        \item \textbf{Model Deployment:} Deploy the models on a secure platform to be used by clinicians and patients.
                    \end{enumerate}
                \item \textbf{Libraries:} Matplotlib, Pandas, NumPy, Scikit-learn, TensorFlow, Keras, PyTorch, Flask, Django
                \item \textbf{Manual Work:}
                    \begin{enumerate}
                        \item Building own model?
                        \item Collection of dataset of diseases and percent chance of symptoms
                    \end{enumerate}
            \end{enumerate}
    \end{enumerate}
% ------------------------------------------------------------------------------
\end{document}
% ------------------------------------------------------------------------------